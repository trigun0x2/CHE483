% CHE473 Status Report #1
\documentclass{article}
	\usepackage{tabularx}
	\addtolength{\oddsidemargin}{-.875in}
	\addtolength{\evensidemargin}{-.875in}
	\addtolength{\textwidth}{1.75in}
	\addtolength{\topmargin}{-.875in}
	\addtolength{\textheight}{1.75in}
	\linespread{2}

\begin{document}
	\title{CHE483 Status Report 2 \\ Gas Diffusion Layer Loading for Breathalyser Performance}
	\author{Group 40: \\James Zhang \\ Jeffrey Tong \\ Vince Cheung \\ Yang Guo }
	\date{Last Updated: \today}
	\maketitle
	\cleardoublepage
		\tableofcontents
\pagebreak
	\section{Current Progress}
	At this time, all laboratory equipment required to create and test the PEMFC breathalysers have been purchased, set up, and tested. Calculations pertaining to the catalyst loading have been completed. All Nafion membranes were pre-treated, catalyst ink at various loadings have been prepared and sprayed. The spraying efficiencies/recovery ratios have been calculated and deemed reasonable. After thoroughly drying all prepared GDLs, surfaces were deemed suitably smooth and even for use in sensor assemblies. Preparations are under way to perform hot pressing and create the final sensors for further testing. This is expected to be completed by the end of this week so that sensor testing can be completed with the breathalyser simulators by the end of next week. Due to time constraint, changes to the platinum loading of anode and cathode will be made simultaneously, to reduce the need to produce more catalyst ink containing varying amounts of platinum and the total number of trials to be performed. Although if data analysis from breathalyser testing show the need for additional testing, a limited number of additional sensors may be produced. \\
	Testing apparatus previously used by other members of Professor Chen's team has been vacated and will be available for use for the foreseeable future. Testing is expected to begin with initial blank runs to establish normal operation and analysis of results will proceed alongside data acquisition so that any additional as well as repeat trials can be performed as necessary. Overall progress is on track to finish the experimental portion of the project by the first week of March.\\
	Preliminary ethanol/water thermodynamics modelling has been completed using Aspen Plus, but further model refinements as well as repeat calculations using other models may be completed in MATLAB within the next week, to verify breathalyser testing results.\\
	Experimental procedures have been compiled and updated. Preliminary layout of the poster is ongoing and the overall layout is will be finalized by the end of this week. Images and basic content including experimental design/set-up, background some theoretical information have been compiled for the poster and will be edited and inserted by next week. Some experimental results are available but final results will be added as soon as they are available.
\pagebreak
	\section{Executive Summary}
	The typical fuel cell used in a breathalyser is a Polymer Electrolyte Membrane Fuel Cell, or PEMFC. In this fuel cell, hydrogen gas is split into hydrogen ions, the chemical potential of which is used to generate electricity. This current can then be used to calculate the concentration of ethanol in the breath sample. In a breathalyser, the hydrogen comes from the degradation of ethanol to acetaldehyde and hydrogen ions, which is done over a catalyst known as platinum black; platinum particles finely dusted with carbon atoms. Creation and application of this catalyst contribute a major portion of both time and cost of breathalyser production.
Not including the actual processing and application costs, the price of the high-purity platinum used in the sensors is already \$500 USD per gram. Furthermore, the application of said platinum-black catalyst is a lengthy process, due to the even coating requirements for accurate sensor performance.  As such, any opportunities to reduce the amount of catalyst required and/or reduce the application time will be highly important in refining the production process. This project aims to do the former, and as such reduce the application time required by reducing the total amount of catalyst applied.\\
	In fuel cell ethanol sensors, the platinum-black catalyst is applied to the Gas Diffusion Layer, or GDL, which is on both sides of the fuel cell membrane. The amount of catalyst present is expressed as “catalyst loading,” with units of mg/cm\textsuperscript{2}. The goal of this project is to optimize this catalyst loading. Key performance parameters that will be monitored include: ambient temperature, ambient humidity, breath input, input BAC, sensor accuracy, sensor start-up time, and required cool-down between tests. The initial experiment performed is a 22 statistical experiment in catalyst loading on both the sensor anode and cathode, on which linear regression is performed to determine further experimental direction. This is to expedite the experimental process given the length of time needed to fabricate sensors. Accuracy will mostly be tested at 100 BAC, until a higher level of detail is required. \\
	The primary deliverable of this project is the reduction in catalyst usage in breathalyser sensors, and additional accuracy of the sensors themselves. Further benefits can be expected in the reduction of catalyst creation and application time. Health and safety concerns in this project include dangers of working with very fine platinum-black catalyst nano-particulate, and other routine lab safety concerns. The environmental impact of this project is mostly expected to come from the reduction in use of platinum catalyst, and the reduction of process time required in catalyst application.
\pagebreak
	\section{Introduction}
	Presently, ethanol, found in alcoholic beverages, is the most-sold legally-obtainable recreational drug. Its consumption leads to a large variety of effects typically seen in depressants, such as disorientation, loss of motor skills, unconsciousness and death at higher dosages. A large number of alcohol-related incidents and injuries involve the operation of a motor vehicle after consumption, despite there being laws against doing so. As such, strict monitoring practices are required for the consumption of alcohol and usage of vehicles. The most common of these practices is the road-side examination. While various methods have been employed in the past, such as behavioural tests and questioning on the part of police officers, these have proven to be ineffective at actually determining a person’s level of intoxication, and are not considered admissible as legal evidence. \\
	One solution to this is the portable breathalyser. A breathalyser is a device that measures the concentration of alcohol in a person’s blood via a breath sample and various chemical methods. This bypasses concerns such as human error and the rapid denaturing of alcohol in a person’s blood, which interfere with other methods such as officers’ personal judgement and in-station blood testing. Furthermore, with high accuracy methods, breathalyser readings can be admitted as evidence in-court, simplifying rulings on inebriated driving. Current breathalyser technologies are divided into three categories: Firstly, semi-conductor breathalysers, which use the change of voltage across a semi-conductor based on ethanol concentration in the air to detect the concentration of alcohol in a person’s breath. Secondly, infrared breathalysers, which detect the quantity of a wavelength of light that a sample of breath absorbs in order to determine the amount of alcohol present. Finally, fuel cell breathalysers detect alcohol based on the current generated from the degradation of ethanol to acetaldehyde and a hydrogen ion which moves through an electrolytic membrane. Semi-conductor and infrared breathalysers each have issues with accuracy and portability, respectively, which leaves fuel cell breathalysers as the prime candidate as an accurate, portable breathalyser.\\
	This project will focus on polymer electrolyte fuel cells (PEMFC). PEMFC performance is greatly influenced by the platinum catalyst used in the oxidization of hydrogen gas. Platinum is the most expensive part of the PEMFC, therefore reducing the amount of catalyst needed in a breathalyser will lower the cost of production greatly.
	\section{Needs Analysis}
		\subsection{Design Problem}
	Breathalysers are hand-held, portable devices that are used to measure a person's blood alcohol content from a breath sample. Of the present technologies available, the most promising is that of fuel cell sensors, which use an electrochemical reaction of alcohol on platinum catalyst to generate a current, which can then be measured to determine the person's level of intoxication. One of the issues facing this path is that of catalyst loading in the gas diffusion layer of the fuel cell. The catalyst used, platinum black, is expensive even in small quantities, and the optimum application of the catalyst for this usage of fuel cell technology is still unclear. As such, the current loading of catalyst in fuel cell alcohol sensors is an area that is largely unstudied.\\
	The overall goal for the project is to decrease the loading of platinum within the catalyst coating the gas diffusion layer while maintaining or improving breathalyser fuel cell sensor accuracy. Increasing the life time of the breathalyser as well as reducing the time required for warm up and cool down between tests are secondary goals.
		\subsection{Constraints}
			\subsubsection{Weight and Size}
			\subsubsection{Cost}
	\section{Literature Review}
	\section{Project Description}
		\textbf{Project Description:}
		The primary deliverable of this project is a design for a PEMFC to be used in commercial breathalysers which reduces the amount of platinum used while maintaining or increasing their accuracy. Further benefits can be expected in the reduction of catalyst creation and application time. This will be accomplished by creating PEMFCs with varying amounts of platinum catalyst and testing them in a simulated breathalyser set-up to determine accuracy. Health and safety concerns in this project include inhalation risks associated with working with platinum-black catalyst particulate, and other lab safety. The environmental impact of this project is mostly expected to come from the reduction in use of platinum catalyst, and the reduction of processing time required in catalyst application.
	\section{Methods}
		\subsection{Sensor Fabrication}
		\subsubsection{Nafion Pre-treatment}
			\begin{enumerate}
			\item Cut appropriately sized section of Nafion membrane
			\item Boil membrane in 3\% aqueous solution of $H_{2}O_{2}$ for 1 hour until the membrane changes colour from yellow to clear
			\item Rinse with deionized water 3 times and boil in deionized water for 1 hour
			\item Boil membrane in 1mol/L $H_{2}SO_{4}$ for 1 hour
			\item Rinse with deionized water 3 times and boil in deionized water for 1 hour
			\end{enumerate}
		\subsubsection{Catalyst Ink}
			\begin{enumerate}
			\item Determine the desired loading of catalyst, calculate to determine mass of catalyst required
			\item Determine the amount of Nafion solution required
			\item Dilute liquid Nafion to lower concentration for more precise volumetric measurements as necessary
			\item Weigh the required platinum black catalyst
			\item Wet powdered catalyst with one drop of deionized water
			\item Add 2-propanol to the flask containing the catalyst, followed by the appropriately concentrated Nafion solution.
			\item Seal the flask top with Parafilm to prevent contamination of or by the sonicator
			\item Sonicate for 2 to 4 hours
			\end{enumerate}
		\subsubsection{Catalyst Spraying}
			\begin{enumerate}
		\item Sonicate spray solution for two hours 2.5 hours with majority of the vessel submerged. Use a foam floater to keep the entire vessel from submerging.
        \item Retrieve the spray gun case and submerge only the gun in a beaker with IPA or ethanol solution. Make sure the flask attachment is submerged within the solution. Next submerge the beaker within the sonicator and either use a float floater or a clamp to keep the sonication liquid from spilling into the beaker. The sonication should last for about 30 minutes.
        \item While the sonication is happening, prepare the hotplate for spraying. First cut out an aluminium foil sheet which can cover the top of the hotplate used. Carefully cover the top of the hotplate making sure that the aluminium foil near the center of the hotplate is as flat as possible.
        \item Cut out a piece of GDL as required from a large sheet of GDL. For most of these experiments a 8x4cm piece is used. Next, make sure to weigh and record the weight of the GDL for future calculations.
        \item Tape the GDL onto the center hotplate which should now be covered with aluminium foil. Make sure to only use the tape on the very corners of the GDL to allow for maximum spraying area.
        \item Once the GDL is secured onto the hotplate, stand the hotplate up vertically and use any material necessary to balance it.
        \item Turn on the hotplate and set it to $(80-90)^{\circ}$C.
        \item Carefully retrieve the solution and spray gun from the sonicator and make sure the spray gun is dried properly.
        \item Attach the air outlet to the spray gun and fasten with a zip-tie if required.
        \item Attach the loading capsule onto the spray gun and add IPA into the capsule for calibration sprays.
        \item Using the knobs for both the loading speed and air speed adjust the spray gun so that the spray is focused is a small arena and at a low flow rate.
        \item Add the sonicated catalyst solution into the loading capsule.
        \item Point the spray gun towards the GDL and spray in a slow and steady page. Make sure not to stay in any particular spot for a prolonged period of time.
        \item Using the knobs for both the loading speed and air speed adjust the spray gun so that the spray is focused in a small area and at a low flow rate.
        \item Add the sonicated catalyst solution into the loading capsule.
        \item Point the spray gun towards the GDL and spray in a slow and steady page. Make sure not to stay in any particular spot for a prolonged period of time.
        \item If the spray gun fails to spray and the loading capsule still contains solution, wiggle the air flow control to dislodge any large particles and continue spraying.
        \item When all of the solution in the loading capsule has been sprayed, fill the loading capsule three quarters of the way with IPA and continue spraying.
        \item When the IPA solution has been finished, disassemble the spray gun and submerge all parts in the IPA and sonicate again for 20 to 30 minutes.
        \item Turn off the hotplate and carefully remove the GDL to make sure the tape does not rip off the edges of the GDL.
        \item Weigh and record the weight of the GDL again and calculate the loading.
        \item Clean up the lab and dispose of any extra IPA properly
			\end{enumerate}
\pagebreak
\section{Health, Safety and Environment}
\section{Economic Analysis}
\section{Life Cycle Analysis}
\pagebreak
\section{Results and Discussion}
	\subsection{GDL Preparation}
	The designed catalyst loading and GDL spraying results were as below:\\
Spray solutions were created at approximately 60 percent excess catalyst and Nafion loading, in order to compensate for losses during the spraying process. This is why the final loading value and overall recovery ratio for the 1 mg/cm\textsuperscript{2} are greater than 1 and 100 percent respectively.

\begin{table}[ht]
\caption{Theoretical and Actual Catalyst Loading in GDL Preparation}
\centering
\noindent\begin{tabularx}{\linewidth}{X|X|X|X|X|X}
\hline
Theoretical Loading (mg/cm\textsuperscript{2}) & GDL Initial Mass (g) & GDL Final Mass (g) & Spray Efficiency/Recovery Ratio & Actual Loading (mg/cm\textsuperscript{2}) & Overall Recovery Ratio \\
\hline
0.25 & 0.09291 & 0.1004 & 0.475 & 0.198 & 0.79\\
\hline
0.5 & 0.09182 & 0.10377 & 0.379 & 0.316 & 0.632\\
\hline
0.75 & 0.08539 & 0.10318 & 0.376 & 0.470 & 0.627\\
\hline
1.00 & 0.08680 & 0.12543 & 0.613 & 1.021 & 1.021\\
\hline
\end{tabularx}
\label{table:gdlload}
\end{table}
Spray efficiency/recovery ratio is the ratio of total loading achieved to total possible loading achievable with the created spray solution, while overall recovery ratio is the ratio of total loading achieved to the intended theoretical loading. Spray efficiency/recovery ratio is a better used as an indicator of equipment and procedural efficiency, whereas the overall recovery ratio is the primary criteria, as it determines the usability of the GDL. Typically, acceptable minimum overall recovery ratios range from 0.5-0.6 due to losses during spraying and particle retainment in the spray gun. This is in order to ensure sufficient catalyst loading even distribution of the catalyst on the GDL. As such, all four GDLs created thus far have been deemed valid for fuel cell creation.
  	\subsection{Expected Results}
    	\subsubsection{Catalyst Loading}
      After each GDL has been created at their specified catalysts loadings, they are then combined with the electrolytic membrane via hot press to create the fuel cell sensors. These sensors are then tested via industry-standard procedures in order to create current-time curves for analysis. Major points that will be investigated are the peak time, and the recovery time. The curve peak is the highest current output from a sample, and is used to calculate the BAC of the sample. As such, the peak time is directly correlated to the waiting time between the input of a sample to the breathalyser and the earliest possible time to calculate the BAC. Recovery time refers to the minimum amount of time between the end of a sample and the start of a new sample that is required in order to maintain reading accuracy. This time is presently a major point of contention in regulation and the use of breathalysers for roadside BAC checks, due to the desire for multiple readings to ensure accuracy for use as evidence in court.\\
      For peak time, it is expected that peak time will decrease as catalyst loading increases, and vice-versa. This is since the reaction is initially limited by mass transfer between the flow boundary and the catalyst surface for reaction. As such, by increasing or decreasing the quantity of catalyst, and therefore the catalyst surface area, available, it will be possible to shorten or lengthen the peak time respectively. Since the curve peak and peak time are integral parts of the calculation of the BAC, it will be necessary to balance the catalyst loading accordingly in order to maintain both accuracy and testing time.\\
      Overall, changing the catalyst loading of the GDL should not have a large influence on the recovery time of the breathalyser. This is since the limiting factor influencing recovery time is the mass transfer of alcohol and secondary reaction products to and away from the GDL. As such, the quantity of catalyst sites available for reaction should not have an influence on breathalyser recovery time.
    \section{Conclusions}
	\section{Recommendations}
	\section{Bibliography}
	\section{Appendix A: Sample Calculations}
\end{document}