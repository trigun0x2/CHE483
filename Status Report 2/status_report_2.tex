% CHE473 Status Report #1
\documentclass{article}
	\addtolength{\oddsidemargin}{-.875in}
	\addtolength{\evensidemargin}{-.875in}
	\addtolength{\textwidth}{1.75in}
	\addtolength{\topmargin}{-.875in}
	\addtolength{\textheight}{1.75in}
	\linespread{2}

\begin{document}
	\title{CHE483 Status Report 1}
	\author{Group 40: \\James Zhang \\ Jeffrey Tong \\ Vince Cheung \\ Yang Guo }
	\date{Last Updated: \today}
	\maketitle
	\cleardoublepage
	\section{Group Members}
		\begin{itemize}
			\item	James Zhang: 20334659
			\item	Jeffrey Tong: 20336475
			\item	Vince Cheung: 20336382
			\item	Yang Guo: 20329358
		\end{itemize}
	\section{Faculty Mentors}
		\begin{itemize}
			\item Supervising Professor: Zhongwei Chen (zhwchen@uwaterloo.ca)
			\item Main Supervisor (No longer in Waterloo): Siamak Farhad (sfarhad@uwaterloo.ca)
			\item Laboratory Supervisor: Gaopeng Jiang (g2jiang@uwaterloo.ca)
		\end{itemize}
	\section{Project Description}
		\textbf{Project Title:} Gas Diffusion Layer Loading for Breathalyser Performance \\ \\
		\textbf{Project Description: }
		Typical breathalysers currently use technologies such as semi-conductors, infrared and polymer electrolyte fuel cells. This project will focus on polymer electrolyte fuel cells (PEMFC). PEMFC performance is greatly influenced by the platinum catalyst used in the oxidization of hydrogen gas. Platinum is the most expensive part of the PEMFC, therefore reducing the amount of catalyst needed in a breathalyser will lower the cost of production greatly.\\ \\
		The primary deliverable of this project is a design for a PEMFC to be used in commercial breathalysers which reduces the amount of platinum used while maintaining or increasing their accuracy. Further benefits can be expected in the reduction of catalyst creation and application time. This will be accomplished by creating PEMFCs with varying amounts of platinum catalyst and testing them in a simulated breathalyser set-up to determine accuracy. Health and safety concerns in this project include inhalation risks associated with working with platinum-black catalyst particulate, and other lab safety. The environmental impact of this project is mostly expected to come from the reduction in use of platinum catalyst, and the reduction of processing time required in catalyst application.
	\section{Design Problem}
		Breathalysers are hand-held, portable devices that are used to measure a person's blood alcohol content from a breath sample. Of the present technologies available, the most promising is that of fuel cell sensors, which use an electrochemical reaction of alcohol on platinum catalyst to generate a current, which can then be measured to determine the person's level of intoxication. One of the issues facing this path is that of catalyst loading in the gas diffusion layer of the fuel cell. The catalyst used, platinum black, is expensive even in small quantities, and the optimum application of the catalyst for this usage of fuel cell technology is still unclear. As such, the current loading of catalyst in fuel cell alcohol sensors is an area that is largely unstudied. The goal of this project is to expand that area of knowledge, and to optimize the loading of catalyst within the fuel cell gas diffusion layer.
	\section{Current Progress}
		Completed items have been highlighted in the attached Gantt chart.
		At this time, all laboratory equipment required to create and test the PEMFC breathalysers have been purchased, set up, and tested. Calculations pertaining to the catalyst loading have been completed. One membrane has been prepared, but deemed unsuitable for testing due to uneven coating.
		At the beginning of the current term the previous project supervisor, Siamak Farhad, left the University of Waterloo. Previously he was responsible for purchase of equipment, laboratory set-up as well as liaison with Professor Zhongwei Chen and other members of his team. We have therefore contacted Gaopeng Jiang, who had previously been involved in laboratory manufacturing and testing. After reviewing progress made on the project over the past 8 months as well as possible problems brought forth by this setback, some of the project dates have been changed to better reflect current and future plans. This is shown in the attached updated Gantt chart. Literature review is an ongoing process and information gathered is expected to provide insight in analyzing data from the breathalyser tests.
		Currently part of the testing apparatus previously purchased and set-up is being used by other members of Professor Chen's team on another, unrelated breathalyser project. This is expected to continue until the end of January.
	\section{Future Plans and Milestones}
		\subsection{PEMFC Catalyst Preparation}
			Due to the length of the time (approximately 6-8 hours) required to prepare various pieces of the fuel cell membrane including the Nafion membrane, platinum black solution, etc., it is important to begin this process and create as many usable gas diffusion layers and full fuel cell sensors as possible before the testing apparatus is ready for use in February. As such laboratory work has begun and will continue several times a week over the next few weeks.
			Due to problems with previous hot press machines, there is a limited period of time during which the last step of membrane preparation can be performed, however if several samples of platinum black-coated dry GDL and Nafion membrane are prepared, the membrane layers can be assembled in bulk, reducing the amount of time required. As such, the primary task in terms of time will be the actual spraying of the gas diffusion layers. This is due to the availability of the spray gun, along with the slow spraying speed required to ensure an even layer of catalyst on the gas diffusion layer.

		\subsection{Breathalyser Testing}
			Testing of the prepared membrane layers will begin once all the samples containing various amounts of platinum black are ready. The membrane layers will be tested in the breathalyser simulators at various breath alcohol levels, temperatures and air flow rates as outlined in the experiment report. Due to overlapping equipment with other groups in the lab for breathalyser testing as mentioned above, major testing likely will not occur until the end of January.
			Experimental work is expected to conclude by the end of February, although if repeat tests are deemed necessary this may extend the experimental period by a week. Testing of standard membranes will also be conducted to have a reference point for our research.

		\subsection{Literature Review, Final Report and Poster}
			Introduction, experimental set up and theoretical background portions of the final report and poster are being compiled and edited at the moment, with experimental results, conclusions and recommendations to follow as soon as they are available. Literature review is ongoing and expected to continue until the end of experimental work. In particular, testing will be coordinated with the other breathalyser groups in the lab, in order to ensure optimal usage of data and testing.
			Content of the poster is expected to be finalized by mid-March, at which time the final report should also be ready for final editing. A due date for the report has yet to be set but should be determined by the end of this week.
	\section{Procedure}
		\subsection{Catalyst Spraying}
			The following is the procedure used to spray the prepared catalyst solution onto the nafion gas diffusion layer.
			\begin{enumerate}
				\item Sonicate spray solution for two hours 2.5 hours with marjority of the vessel submerged. Use a foam floater to keep the entire vessel from submerging.
        \item Retrieve the spray gun case and submerge only the gun in a beaker with IPA or ethanol solution. Make sure the flask attatchment is submerged within the solution. Next submerge the beaker within the sonicator and either use a float floater or a clamp to keep the sonication liquid from spilling into the beaker. The sonication should last for about 30 minutes.
        \item While the sonication is happening, prepare the hotplate for spraying. First cut out an aluminum foil sheet which can cover the top of the hotplate used. Carefully cover the top of the hotplate making sure that the aluminum foil near the center of the hotplate is as flat as possible.
        \item Cut out a piece of GDL as required from a large sheet of GDL. For most of these experiments a 8x4cm piece is used. Next, make sure to weigh and record the weight of the GDL for future calculations.
        \item Tape the GDL onto the center hotplate which should now be covered with aluminum foil. Make sure to only use the tape on the very corners of the GDL to allow for maximum spraying area.
        \item Once the GDL is secured onto the hotplate, stand the hotplate up vertically and use any material necessary to balance it.
        \item Turn on the hotplate and set it to $(80-90)^{\circ}$ C.
        \item Carefully retrieve the solution and spray gun from the sonicator and make sure the spray gun is dried properly.
        \item Attach the air outlet to the spray gun and fasten with a zip-tie if required.
        \item Attach the loading capsule onto the spray gun and add IPA into the capsule for calibration sprays.
        \item Using the knobs for both the loading speed and air speed adjust the spray gun so that the spray is focused is a small arena and at a low flow rate.
        \item Add the sonicated calatyst solution into the loading capsule.
        \item Point the spray gun towards the GDL and spray in a slow and steady page. Make sure not to stay in any particular spot for a prolonged period of time.
        \item sprays.
        \item Using the knobs for both the loading speed and air speed adjust the spray gun so that the spray is focused is a small arena and at a low flow rate.
        \item Add the sonicated calatyst solution into the loading capsule.
        \item Point the spray gun towards the GDL and spray in a slow and steady page. Make sure not to stay in any particular spot for a prolonged period of time.
        \item If the spray gun fails to spray and the loading capsule still contains solution, wiggle the air flow control to dislodge any large particles and continue spraying.
        \item When all of the solution in the loading capsule has been sprayed, fill the loading capsule three quaters of the way with IPA and continue spraying.
        \item When the IPA solution has been finished, disassemble the spray gun and submerge all parts in the IPA and sonicate again for 20 to 30 minutes.
        \item Turn off the hotplate and carefully remove the GDL to make sure the take does not rip off the edges of the GDL.
        \item Weigh and record the weight of the GDL again and calculate the loading.
        \item Clean up the lab and dispose of any extra IPA properly
			\end{enumerate}
  \section{Expected Results}
    \subsection{Catalyst Loading}
      After each GDL has been created at their specified catalysts loadings, they are then combined with the electrolytic membrane via hot press to create the fuel cell sensors. These sensors are then tested via industry-standard procedures in order to create current-time curves for analysis. Major points that will be investigated are the peak time, and the recovery time. The curve peak is the highest current output from a sample, and is used to calculate the BAC of the sample. As such, the peak time is directly correlated to the waiting time between the input of a sample to the breathalyser and the earliest possible time to calculate the BAC. Recovery time refers to the minimum amount of time between the end of a sample and the start of a new sample that is required in order to maintain reading accuracy. This time is presently a major point of contention in regulation and the use of breathalysers for roadside BAC checks, due to the desire for multiple readings to ensure accuracy for use as evidence in court.\\

      For peak time, it is expected that peak time will decrease as catalyst loading increases, and vice-versa. This is since the reaction is initially limited by mass transfer between the flow boundary and the catalyst surface for reaction. As such, by increasing or decreasing the quantity of catalyst, and therefore the catalyst surface area, available, it will be possible to shorten or lengthen the peak time respectively. Since the curve peak and peak time are integral parts of the calculation of the BAC, it will be necessary to balance the catalyst loading accordingly in order to maintain both accuracy and testing time.\\
      Overall, changing the catalyst loading of the GDL should not have a large influence on the recovery time of the breathalyser. This is since the limiting factor influencing recovery time is the mass transfer of alcohol and secondary reaction products to and away from the GDL. As such, the quantity of catalyst sites available for reaction should not have an influence on breathalyser recovery time.
      \subsection{Humidity and Temperature}
      JAMES ZHANG
\end{document}
